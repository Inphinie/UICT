\documentclass[12pt]{article}

\usepackage{amsmath, amssymb, amsfonts}
\usepackage{graphicx}
\usepackage{hyperref}
\usepackage{physics}
\usepackage{bm}
\usepackage{geometry}
\geometry{margin=1in}

\title{\textbf{The Dirac Equation as an Information-Compression Law:\\
A Unified Framework Linking Fisher Information, Quantum Cellular Automata,\\
and Relativistic Quantum Information}}

\author{Bryan Ouellette\\
\small Independent Researcher}

\date{2026}

\begin{document}

\maketitle

\begin{abstract}
We propose that the Dirac equation is not merely a relativistic wave equation for spin-$\frac{1}{2}$ particles, but the universal dynamical law governing the flow, compression, and redistribution of information in spacetime. We show that three independent information-theoretic derivations of the Dirac equation---Fisher Information (EPI), Quantum Cellular Automata (QCA), and Relativistic Quantum Information (RQI)---are not coincidental parallels but projections of a single generative principle: the Unified Information Compression Theory (UICT).

UICT postulates that physical systems evolve to maximize contextual coherence per unit entropy:
\[
\Psi^* = \arg\max_\Psi \frac{C(\Psi|\Omega)}{H(\Psi)+\epsilon}.
\]
We demonstrate that the Dirac equation emerges as the Euler--Lagrange equation of this optimization, and that mass, spin, chirality, and particle--antiparticle duality correspond to distinct modes of information compression. We further show that the $\kappa$-compression parameter introduced in UICT provides a natural interpretation of inertial mass, decoherence rates, and spin--momentum entanglement.

This framework unifies measurement, computation, and entanglement under a single informational ontology, suggesting that the Dirac equation is the fundamental algorithm of physical reality.
\end{abstract}

\section{Introduction}

The Dirac equation has traditionally been interpreted as a relativistic quantum wave equation describing spin-$\frac{1}{2}$ particles. However, three independent research programs have revealed deep information-theoretic structures underlying its form:

\begin{enumerate}
    \item Fisher Information / EPI (Frieden, Yahalom)
    \item Quantum Cellular Automata (D'Ariano, Perinotti, Bisio)
    \item Relativistic Quantum Information (Bittencourt, Peres, Alsing)
\end{enumerate}

UICT provides the missing link unifying these perspectives.

\section{The Unified Information Compression Principle}

UICT postulates that all physical systems evolve to maximize:
\[
\mathcal{S}(\Psi) = \frac{C(\Psi|\Omega)}{H(\Psi)+\epsilon},
\]
where $C$ is contextual compressibility and $H$ is informational entropy.

The stationary action principle:
\[
\delta \mathcal{S} = 0
\]
yields the dynamical equations of the system. We show that applying this principle to a spinor-valued field yields the Dirac equation.

\section{Fisher Information and the EPI Derivation}

Frieden's EPI principle states:
\[
\delta(I - J) = 0,
\]
where $I$ is Fisher information acquired by measurement and $J$ is information bound in the system.

UICT identifies:
\[
I \equiv C(\Psi|\Omega), \qquad J \equiv H(\Psi).
\]

Taking the variation of:
\[
\ln\left(\frac{C}{H}\right)
\]
yields the same extremal condition as EPI.

The Euler--Lagrange equation of this functional is the Dirac equation.

\section{Quantum Cellular Automata and Information Flow}

D'Ariano's QCA program derives Dirac from unitarity, locality, homogeneity, and isotropy. In this framework:

\begin{itemize}
    \item information flows at speed $c$,
    \item mass arises from coupling between left/right chiral modes,
    \item the Dirac equation is the continuum limit of a reversible computation.
\end{itemize}

UICT interprets this coupling as recursive compression:
\[
m \propto \kappa^n.
\]

\section{Relativistic Quantum Information and Spinor Entanglement}

RQI shows that a Dirac spinor is equivalent to a two-qubit system:
\[
H_{Dirac} = H_{spin} \otimes H_{parity}.
\]

Boosts induce Wigner rotations, creating spin--momentum entanglement.

UICT interprets this as cross-modal compression redistribution:
\[
H_{\text{cross}} \leftrightarrow S_{\text{ent}}.
\]

\section{The $\kappa$-Compression Law for Mass}

UICT proposes:
\[
m = m_P \, \kappa^n,
\]
with $n \approx 33$--$43$ depending on particle type.

This law reproduces:
\begin{itemize}
    \item lepton mass hierarchy,
    \item quark mass hierarchy,
    \item hadron mass scaling.
\end{itemize}

It also predicts:
\[
\Gamma_{dec} \propto \kappa^n, \qquad
\theta_W \propto v \ln(\kappa), \qquad
\Delta x \propto \frac{1}{\kappa}.
\]

\section{Experimental Predictions}

\subsection{Fisher Information Scaling}
\[
\Delta x \propto \frac{1}{\kappa}.
\]

\subsection{Decoherence--Mass Law}
\[
\Gamma_{dec} \propto \kappa^n.
\]

\subsection{Wigner Rotation Scaling}
\[
\theta_W \propto v \ln(\kappa).
\]

\subsection{Compression-Dependent Gravity}
\[
G_{diamond} - G_{amorphous} \sim 10^{-7}.
\]

\section{Discussion}

UICT reveals that:
\begin{itemize}
    \item EPI describes measurement of information,
    \item QCA describes flow of information,
    \item RQI describes storage and entanglement of information,
    \item Dirac describes optimization of information.
\end{itemize}

These are not separate theories but facets of a single operator.

\section{Conclusion}

We argue that the Dirac equation is the universal algorithm governing the compression, flow, and redistribution of information in spacetime. UICT provides the ontological foundation unifying Fisher Information, QCA, and RQI, revealing mass, spin, and chirality as emergent properties of informational compression.

\end{document}
